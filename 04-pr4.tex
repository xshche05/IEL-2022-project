\section{Příklad 4}
% Jako parametr zadejte skupinu (A-H)
\ctvrtyZadani{A}
\makebox[\linewidth]{\rule{\textwidth}{0.5pt}}
\subsection{Řešení}
Určíme směr proudu pro každou smyčku, určíme proudy $I_A$, $I_B$, $I_C$ a jích směr:
\begin{center}
\begin{circuitikz}[european resistors]
\draw
(0, 4) to[sV>=$u_1$, invert] (6, 4) to (6, 0)
(0, 4) to[R=$R_1$, -*] (0, 2)
to [L, label = $L_1$] (2, 2)
to [R=$R_2$] (4, 2)
to [L, label = $L_2$, *-*] (6, 2)
(6, 0) to[sV>=$u_2$, -*] (4, 0)
(4, 0) to[C=$C_1$] (4, 2)
(4, 0) to[C=$C_2$, v>=$ $] (0, 0) to (0, 2)
(4, 0) to[short, i=$i_{C_2}$] (1.5, 0)
(3,3) node[circulator,scale=0.6]{}
(2,1) node[circulator,scale=0.6]{}
(5,1) node[circulator,scale=0.6]{}
[anchor=south west]
(3.1,3) node{$I_{A}$}
(2.1,1) node{$I_{B}$}
(5.1,1) node {$I_{C}$}
;
\end{circuitikz}
\end{center}
Spočítáme hodnoty impedance pro cívky a kondenzátory:
\begin{gather*}
\omega = 2\pi f = 2*\pi*70 = 140\pi = 439.8230 Rad \\
Z_{C_1} = \frac{-j}{\omega C_1} = \frac{-j}{439.8230 * 200*10^{-6}} = -11.3682j \Omega\\
Z_{C_2} = \frac{-j}{\omega C_2} = \frac{-j}{439.8230 * 105*10^{-6}} = -21.6537j \Omega\\
Z_{L_1} = j \omega L_1 = j * 439.8230 * 120 * 10^{-3} = 52.7788j \Omega\\
Z_{L_2} = j \omega L_2 = j * 439.8230 * 100 * 10^{-3} = 43.9823j \Omega\\
\end{gather*}
Spočítáme napětí $u_1$ a $u_2$:
\begin{gather*}
u_1 = U_1 * \sin{(2\pi ft)} = U_1 * \sin{(2\pi f * \frac{\pi}{4 \pi f})} = U_1 * \sin{(\frac{\pi}{2})} = U_1 * 1 = 3\Vo\\
u_2 = U_2 * \sin{(2\pi ft)} = U_2 * \sin{(2\pi f * \frac{\pi}{4 \pi f})} = U_2 * \sin{(\frac{\pi}{2})} = U_2 * 1 = 5\Vo
\end{gather*}
Dle II. Kirchhoffova zákona zapíšeme pro každou smyčku rovnice pro napětí, dále dostaneme soustavu rovnic pro napětí ve smyčkách:
\begin{gather*}
I_A: \quad U_{L_2} + U_{R_2} + U_{L_1} + U_{R_1} + u_1 = 0 \\
I_B: \quad U_{C_1} + U_{C_2} + U_{L_1} + U_{R_2} = 0 \\
I_C: \quad u_2 + U_{C_1} + U_{L_2} = 0
\end{gather*}
Určíme napětí přes proud a odpor:
\begin{gather*}
I_A: \quad I_A*(Z_{L_2} + R_2 + Z_{L_1} + R_1) - I_B*(Z_{L_1} + R_2) - I_C*(Z_{L_2}) = -u_1\\
I_B: \quad -I_A*(Z_{L_1} + R_2) + I_B*(Z_{C_1} + Z_{C_2} + Z_{L_1} + R_2) - I_C*(Z_{C_1}) = 0\\
I_C: \quad -I_A*(Z_{L_2}) - I_B*(Z_{C_1}) + I_C*(Z_{L_2} + Z_{C_1}) = -u_2
\end{gather*}
Matice pro výpočet proudů v každou smyčce: 
\begin{gather*}
\begin{pmatrix}
Z_{L_2}+R_2+Z_{L_1}+R_1 & -Z_{L_1}-R_2 & -Z_{L_2} \\
-Z_{L_1}-R_2 & Z_{C_1}+Z_{C_2}+Z_{L_1}+R_2 & -Z_{C_1} \\
-Z_{L_2} & -Z_{C_1} & Z_{L_2}+Z_{C_1}
\end{pmatrix}
\times
\begin{pmatrix}
I_A\\
I_B\\
I_C
\end{pmatrix}
=
\begin{pmatrix}
-u_1\\
0\\
-u_2
\end{pmatrix}
\end{gather*}
Spočítáme hodnoty proudů v každou smyčce:
\begin{gather*}
\begin{pmatrix}
26.0000+96.7611j & -14.0000-52.7788j & -43.9823 \\
-14.0000-52.7788j & 14.0000+19.7568j & 11.3682 \\
-43.9823 & 11.3682 & 32.6141
\end{pmatrix}
\times
\begin{pmatrix}
I_A\\
I_B\\
I_C
\end{pmatrix}
=
\begin{pmatrix}
-3\\
0\\
-5
\end{pmatrix}
\end{gather*}
\begin{gather*}
I_A = (-0.1025 - 0.1024j) \Am \\
I_B = (-0.0568 - 0.3126j) \Am \\
I_C = (-0.1185 + 0.1242j) \Am
\end{gather*}
\subsection{Výsledky}
Spočítáme proud $i_{C_2}$, napětí $|U_{C_{2}}|$ a fázový posun $\varphi_{C_{2}}$:
\begin{gather*}
i_{C_2} = I_B = (-0.0568 - 0.3126j) \Am \\
U_{C_{2}} = i_{C_2} * Z_{C_2} = (-0.0568 - 0.3126j) * (-21.6537j) = (-6.7697 + 1.2291j) \Vo \\
|U_{C_{2}}| = \sqrt{Re(U_{C_{2}})^2 + Im(U_{C_{2}})^2} = \sqrt{(-6.7697)^2 + (1.2291)^2} = 6.8804 \Vo \\
\varphi_{C_{2}} = \arctan(\frac{Im(U_{C_2})}{Re(U_{C_2})}) + \pi = \arctan(\frac{1.2291}{-6.7697}) + \pi = 2.9620 Rad = 169.7093^{\circ}
\end{gather*}
