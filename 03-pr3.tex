\section{Příklad 3}
% Jako parametr zadejte skupinu (A-H)
\tretiZadani{C}
\makebox[\linewidth]{\rule{\textwidth}{0.5pt}}
\subsection{Řešení}
Dle II. Kirchhoffova zákona zapíšeme rovnice pro uzly $A$ (začátek napětí $U_A$), $B$ (začátek napětí $U_B$), $C$ (začátek napětí $U_C$):
\begin{gather*}
A: \quad I_{R_1} - I_{R_2} - I_{R_3} = 0\\
B: \quad I_{R_2} - I_{R_4} - I_2 = 0\\
C: \quad I_{R_4} + I_2 - I_1 - I_{R_5} = 0
\end{gather*}
Spočítáme proudy pro každý rezistor s napětími $U_A, U_B, U_C$:
\begin{gather*}
I_{R_1} = \frac{U-U_{A}}{R_{1}} \\
I_{R_2} = \frac{U_{A}-U_{B}}{R_{2}} \\
I_{R_3} = \frac{U_{A}}{R_{3}} \\
I_{R_4} = \frac{U_{B}-U_{C}}{R_{4}}\\
I_{R_5} = \frac{U_{C}}{R_{5}}
\end{gather*}
Dostaneme rovnice pro každý uzel, kde proud je vyjádřen napětím a odporem:
\begin{gather*}
A: \quad \frac{110-U_{A}}{44} - \frac{U_{A}-U_{B}}{31} - \frac{U_{A}}{56} = 0 \\
B: \quad \frac{U_{A}-U_{B}}{31} - \frac{U_{B}-U_{C}}{20} - 0.75 = 0 \\
C: \quad \frac{U_{B}-U_{C}}{20} + 0.75 - 0.85 - \frac{U_{C}}{30} = 0
\end{gather*}
Vyneseme koeficienty za závorky:
\begin{gather*}
A: \quad -U_A * \frac{92}{1263} + U_B * \frac{1}{31} = -2.5 \\
B: \quad U_A * \frac{1}{31} - U_B * \frac{51}{620} + U_C * \frac{1}{20} = 0.75\\
C: \quad U_B * \frac{1}{20} - U_C * \frac{1}{12} = 0.1
\end{gather*}
Zapíšeme soustavu rovnic do matice a spočítáme hodnoty $U_A$, $U_B$, $U_C$:
\begin{gather*}
\begin{pmatrix}
-\frac{92}{1263} & \frac{1}{31} & 0\\
\frac{1}{31} & -\frac{51}{620} & \frac{1}{20}\\
0 & \frac{1}{20} & -\frac{1}{12}
\end{pmatrix}
\times
\begin{pmatrix}
U_A\\
U_B\\
U_C
\end{pmatrix}
=
\begin{pmatrix}
-2.5\\
0.75\\
0.1
\end{pmatrix}
\end{gather*}
\begin{gather*}
U_A = 37.7857 \Vo\\
U_B = 7.8245 \Vo\\
U_C = 3.4947 \Vo
\end{gather*}
\noindent
\subsection{Výsledky}
Vypočítáme si $U_{R_4}$, a pak dopočítáme proud $I_{R_4}$:
\begin{gather*}
U_{R_4} = U_B - U_C = 7.8245 - 3.4947 = 4.3298 \Vo \\
I_{R_4} = \frac{U_{R_4}}{R_4} = \frac{4.3298}{20} = 0.2165 \Am
\end{gather*}