\section{Příklad 2}
% Jako parametr zadejte skupinu (A-H)
\druhyZadani{E}
\makebox[\linewidth]{\rule{\textwidth}{0.5pt}}
\subsection{Řešení}
Pro řešení příkladů vytvoříme ekvivalentní obvod a zjistíme $R_i$ a $U_i$:
\begin{center}
\begin{circuitikz}[european resistors, american voltages]
\draw
(0,2) to[V_=$U_{i}$] (0,0)
(0,2) to[R=$R_{i}$] (2,2) to[short, i=$I_{R_5}$] (3, 2);
\ctikzset{voltage=straight}
\draw
(3,2) to[R=$R_5$,  *-*, v=$U_{R_5}$] (3,0) -- (0, 0)
(3, 2) node[anchor=west] {A}
(3, 0) node[anchor=west] {B}
;
\end{circuitikz}
\end{center}
Nejprve v původním obvodu zkratuji (odpojím) napěťový zdroj a odpojím $R_5$, obvod zjednoduším a zjistím $R_i$:
\begin{center}
\begin{circuitikz}[european resistors]
\draw
(0,0) to[R=$R_2$] (2, 0) to[short, *-*] (3, 0)
(2,0) to[R, l_=$R_4$] (2, 2) to[short, *-*] (3, 2)
(2,2) to[R, l_=$R_3$] (2, 4)
(0,4) to[R, l_=$R_1$] (2, 4)
(0,0) -- (0,4)
(3, 2) node[anchor=west] {A}
(3, 0) node[anchor=west] {B}
;
\end{circuitikz}
\end{center}
Rezistory $R_1$, $R_2$ a $R_3$ jsou zapojené sériově - zjednoduším na $R_{123}$:
\begin{gather*}
R_{123} = R_1 + R_2 + R_3 = 150 + 335 + 625 = 1110 \Omega
\end{gather*}
\begin{center}
\begin{circuitikz}[european resistors]
\draw
(0,0) -- (2, 0) to[short, *-*] (3, 0)
(2,0) to[R, l_=$R_4$] (2, 2) to[short, *-*] (3, 2)
(0,0) to[R, l_=$R_{123}$] (0,2) -- (2, 2)
(3, 2) node[anchor=west] {A}
(3, 0) node[anchor=west] {B}
;
\end{circuitikz}
\end{center}
Rezistory $R_{123}$ a $R_4$ jsou zapojené paralelně - zjednoduším a zjistím $R_i$:
\begin{gather*}
R_i = \frac{1}{\frac{1}{R_{123}}+\frac{1}{R_4}} =  \frac{1}{\frac{1}{1110}+\frac{1}{245}} \approx 200.7011 \Omega
\end{gather*}
Nyní je třeba zjistit $I_x$:
\begin{center}
\begin{circuitikz}[european resistors, american voltages]
\draw
(0,0) to[R=$R_2$] (2, 0) to[short, *-*] (3, 0)
(2,0) to[R, l_=$R_4$] (2, 2) to[short, *-*] (3, 2)
(2,2) to[R, l_=$R_3$] (2, 4)
(0,4) to[R, l_=$R_1$] (2, 4)
(0,0) to[V=$U$, invert] (0,4)
(3, 2) node[anchor=west] {A}
(3, 0) node[anchor=west] {B}
;
\end{circuitikz}
\end{center}
\begin{gather*}
R_{1234} = R_1 + R_2 + R_3 + R_4 = 150 + 335 + 625 + 245 = 1355\Omega \\
I_x = \frac{U}{R_{1234}} = \frac{250}{1355} \approx 0.1845 \Am
\end{gather*}
$U_i$ bude se rovnat $U_{R_4}$:
\begin{gather*}
U_i = U_{R_4} = I_x * R_4 = 0.1845 * 245 \approx 45.203 \Vo
\end{gather*}
Teď třeba zjistit $U_{R_5}$ a $I_{R_5}$:
\begin{center}
\begin{circuitikz}[european resistors, american voltages]
\draw
(0,2) to[V_=$U_{i}$] (0,0)
(0,2) to[R=$R_{i}$] (2,2) to[short, i=$I_{R_5}$] (3, 2);
\ctikzset{voltage=straight}
\draw
(3,2) to[R=$R_5$,  *-*, v=$U_{R_5}$] (3,0) -- (0, 0)
(3, 2) node[anchor=west] {A}
(3, 0) node[anchor=west] {B}
;
\end{circuitikz}
\end{center}
\subsection{Výsledky}
$R_i$ a $R_5$ jsou zapojené sériově, spočítáme $U_{R_5}$ a $I_{R_5}$:
\begin{gather*}
U_{R_5} = \frac{R_5}{R_i + R_5} * U_i = \frac{600}{200.7011 + 600} * 45.203 \approx 33.8725 \Vo \\
I_{R_5} = \frac{U_{R_5}}{R_5} = \frac{33.8725}{600} = 0.0565 \Am
\end{gather*}